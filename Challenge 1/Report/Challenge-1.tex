% This is LLNCS.DEM the demonstration file of
% the LaTeX macro package from Springer-Verlag
% for Lecture Notes in Computer Science,
% version 2.4 for LaTeX2e as of 16. April 2010
%
\documentclass{llncs}
%
\usepackage{makeidx}  % allows for indexgeneration
%
\usepackage{graphicx}
\graphicspath{ {images/} }
\begin{document}
%
\frontmatter          % for the preliminaries
%
\pagestyle{headings}  % switches on printing of running heads

\mainmatter              % start of the contributions
%
\title{Anomaly Detection Challenges - Challenge I}
%
\titlerunning{Challenge 1}  % abbreviated title (for running head)
%                                     also used for the TOC unless

%
\author{Hamza Tahir (03670002) \and Muhammad Hamza Usmani (03669506)}
%
\authorrunning{Tahir \and Usmani} % abbreviated author list (for running head)
%
\institute{Technical University of Munich}
%%%% list of authors for the TOC (use if author list has to be modified)

\maketitle              % typeset the title of the contribution


%
\section{Introduction}
%
This brief report serves as a purpose to present and explain the methodologies applied to tackle the first challenge in the Practical: Anomaly Detection Challenges. Section~2 discusses the challenge task and the data set for the machine learning/anomaly detection task. Section~3 explains the approaches adopted for the task.
%
\section{The Challenge}
%
%
\subsubsection{Machine Learning Task}
%
The machine learning task for this challenge is to determine if given soil sample is that of normal soil or a cotton soil. The task is a supervised learning, binary classification with two decision classes:
\begin{table}
\caption{Decision Classes}
\begin{center}
\begin{tabular}{r@{\quad}rl}
\hline
\multicolumn{1}{l}{\rule{2pt}{0pt}
Class}&\multicolumn{2}{l}{Numeric Representation}\\[2pt]
\hline\rule{0pt}{12pt}
Normal Soil&    0& \\
Cotton Soil&     1& \\[2pt]
\hline
\end{tabular}
\end{center}
\end{table}
%
\subsubsection{The Data set}
The dataset of the challenge consisted of a subset of the Satellite Imagery dataset originally published by Ashwin Srinivasan. The training set consists of 4435 samples while the test set consists of 2000 samples. Samples have 36 features, that characterize spatial neighborhood and spectral band characteristics. 70\% of training samples have missing values for randomly chosen feature values.

\section{Methodology}
This section explains the data analysis and the machine learning process to build the model for classifying the given samples as normal  or cotton soil samples. 
\subsection{Data Preprocessing}
The  given dataset has missing values some features in many samples, to build a generic classifier it is therefore important to treat the missing values to build a robust classifier that is relatively generic. The authors used the following techniques to deal with the missing values in the dataset:
%
\subsubsection{Feature Mean:}
%
The missing values were imputed by average of values. Different averaging schemes were adopted, under first scheme as proposed by Runkler \cite{runkler}, the missing value of a particular feature is replaced by average value of of the feature. This averaging technique is generic to replace missing values for all kinds of datasets.
%
\subsubsection{Feature Maximum:}
%
In this technique, missing values can be treated as outliers and thus they can be imputed with maximum possible values \cite{runkler}. Under this scheme, missing values are replaced by maximum value of the feature. This is exactly the same as the Feature Mean technique, only difference is replacing average with the maximum feature value.
%
\subsubsection{Nine-Neighborhood Mean:}
%
Another spatial average based technique used is 9-neighborhood averages, considering the given dataset comprises of four different observations in a nine-neighborhood, missing values were also replaced by averages in a particular nine-neighborhood. 
%
\subsubsection{Immediate Neighborhood Mean:}
%
This approach is similar to the Nine-Neighborhood one but here image locality is leveraged to better replace missing values. The underlying assumption is that closely related pixels have similar values. Therefore, rather than using the entire 9 neighborhood, only 'immediate' neighbors are used for the average value. 'Immediate' neighbors are defined as the pixels that are to the left, right, top and bottom of the 9-neighborhood. 
%
Figure 1. summarizes all imputation techniques:

\begin{figure}
     \centering
    \includegraphics[width=1.0\textwidth]{ReplacementSchemes4}
    \caption{Missing Value Schemes}
    \label{fig:my_label}
\end{figure}

\subsection{Machine Learning Techniques}
Following machine learning techniques were used to classify given samples as "Cotton Soil" or "Normal Soil":
\begin{enumerate}
   \item Naive Bayes
   \item Support Vector Classifier (SVC)
   \item K-Nearest Neighbors (KNN)
   \item Decision Trees
   \item Neural Network - Multi Layer Perceptron
   \item Random Forest
 \end{enumerate}
 
\subsection{Generalizing - Developing a Robust Classifier}
To build a robust classifier, that is relatively general and is not restricted only to the given training set following techniques were used:

\subsubsection{Three-fold Cross Validation:} Three-fold cross validation was used to overcome the problem of over-fitting, and to build a model to that will generalize to an independent dataset, Hawkins et. al. (see \cite{hawkins:eke}).  

\subsubsection{Feature Weighing:} Another technique used to improve accuracy and to generalize the classification model was to weigh the features. Under the assumption that not all features contribute equally to determine the class of a sample, different features can have different weights that represent their relative importance to the classification model. In particular, we could not be certain that a certain spectral band was more or less important for the classification of cotton soil. It might be the fact that the green spectrum was more important, or maybe the infra-red spectrum. 
\newline
Feature weighing is a complex technique. In our model we used basic techniques based upon correlation to determine weights of the features. 
First, we extracted each feature out individually and trained our classifiers on each separately. The feature that had the lowest accuracy after 3-fold validation was then completely removed from the training of the final model. This way, we fine-tuned our model to better estimate the classification.

\section{Results}
Results of the challenge are summarized in this section. We present two basic results. One is with out classifiers trained on all features, without weighing. Second with our classifiers trained with only three features (first, second and fourth), therefore giving 0 weight to the third feature.  

\begin{table}
\centering
  \caption{Best Average Cross-Validation Accuracies (without feature weighing) }
\begin{tabular}{ |p{3cm}||p{2cm}|p{2cm}|p{2cm}| p{2cm}| }
 \hline
 \multicolumn{5}{|c|}{Training Accuracies} \\
 \hline
 Technique                  &Feature Mean    &Feature Max    &Nine-Neighbor  &Immediate-Neighbor\\
 \hline
  Decision Trees            &97.519    &98.528         &96.550         &97.723\\
  Multi-Layer Perceptron    &96.280    &97.768      &97.768     &98.016\\
 Naive Bayes                &98.422    &96.979      &98.535     &98.512\\
 KNN                        &98.858    &96.731      &98.557     &98.783\\
 SVC                        &97.813    &94.295      &97.858     &98.377\\
 Random Forest              &98.647    &97.610      &97.970     &98.715\\
 \hline
 Average     &\textbf{97.796}     &\textbf{96.985}       &\textbf{97.873}     &\textbf{98.354}\\
  \hline
\end{tabular}
\label{table}
\end{table}

\begin{table}
	\centering
	\caption{Best Average Cross-Validation Accuracies (with feature weighing) }
	\begin{tabular}{ |p{3cm}||p{2cm}|p{2cm}|p{2cm}| p{2cm}| }
		\hline
		\multicolumn{5}{|c|}{Training Accuracies} \\
		\hline
		Technique& Feature Mean & Feature Max & Nine-Neighbor & Immediate-Neighbor\\
		\hline
		Decision Trees              &97.994    &97.002      &97.610     &97.566\\
		Multi-Layer Perceptron      &97.250    &95.942      &97.565     &97.632\\
		Naive Bayes                 &98.377    &96.397      &98.557    &98.512\\
		KNN                         &97.881    &96.619      &98.693     &98.85\\
		SVC                         &97.565    &94.656      &98.197     &98.377\\
		Random Forest               &98.625    &98.197      &98.715     &98.625\\
		\hline
		 Average     &\textbf{97.948}     &\textbf{96.468}       &\textbf{98.222}     &\textbf{98.260}\\
		 \hline
	\end{tabular}
	\label{table}
\end{table}

As one might expect, the average methods that leveraged the spatial locality of the image pixels proved to have the highest training accuracy in all the classifiers. Feature weighing on the other hand, had negligible effect on the result of the training accuracies.\\
The best classifier overall proved to be the KNN model, followed closely by the Random Forest approach. This also makes sense, as these classifiers are overall robust to high variance and high dimensional data.

%
% ---- Bibliography ----
%
\begin{thebibliography}{5}
%
\bibitem {runkler}
Runkler, T:
Data Analysis 
Models and Algorithms for Intelligent Data Analysis.
Springer, 24--25 (2011)

\bibitem {hawkins:eke}
Hawkins D. , Basak S. , and Denise M. 
Assessing Model Fit by Cross-Validation
J. Chem. Inf. Comput. Sci., 2003, 43 (2), pp 579–586 (2003)

\end{thebibliography}
\end{document}
